% Options for packages loaded elsewhere
\PassOptionsToPackage{unicode}{hyperref}
\PassOptionsToPackage{hyphens}{url}
%
\documentclass[
]{book}
\usepackage{lmodern}
\usepackage{amssymb,amsmath}
\usepackage{ifxetex,ifluatex}
\ifnum 0\ifxetex 1\fi\ifluatex 1\fi=0 % if pdftex
  \usepackage[T1]{fontenc}
  \usepackage[utf8]{inputenc}
  \usepackage{textcomp} % provide euro and other symbols
\else % if luatex or xetex
  \usepackage{unicode-math}
  \defaultfontfeatures{Scale=MatchLowercase}
  \defaultfontfeatures[\rmfamily]{Ligatures=TeX,Scale=1}
\fi
% Use upquote if available, for straight quotes in verbatim environments
\IfFileExists{upquote.sty}{\usepackage{upquote}}{}
\IfFileExists{microtype.sty}{% use microtype if available
  \usepackage[]{microtype}
  \UseMicrotypeSet[protrusion]{basicmath} % disable protrusion for tt fonts
}{}
\makeatletter
\@ifundefined{KOMAClassName}{% if non-KOMA class
  \IfFileExists{parskip.sty}{%
    \usepackage{parskip}
  }{% else
    \setlength{\parindent}{0pt}
    \setlength{\parskip}{6pt plus 2pt minus 1pt}}
}{% if KOMA class
  \KOMAoptions{parskip=half}}
\makeatother
\usepackage{xcolor}
\IfFileExists{xurl.sty}{\usepackage{xurl}}{} % add URL line breaks if available
\IfFileExists{bookmark.sty}{\usepackage{bookmark}}{\usepackage{hyperref}}
\hypersetup{
  pdftitle={2021 Third Year Phage Hunters},
  pdfauthor={Ben Temperton},
  hidelinks,
  pdfcreator={LaTeX via pandoc}}
\urlstyle{same} % disable monospaced font for URLs
\usepackage{longtable,booktabs}
% Correct order of tables after \paragraph or \subparagraph
\usepackage{etoolbox}
\makeatletter
\patchcmd\longtable{\par}{\if@noskipsec\mbox{}\fi\par}{}{}
\makeatother
% Allow footnotes in longtable head/foot
\IfFileExists{footnotehyper.sty}{\usepackage{footnotehyper}}{\usepackage{footnote}}
\makesavenoteenv{longtable}
\usepackage{graphicx}
\makeatletter
\def\maxwidth{\ifdim\Gin@nat@width>\linewidth\linewidth\else\Gin@nat@width\fi}
\def\maxheight{\ifdim\Gin@nat@height>\textheight\textheight\else\Gin@nat@height\fi}
\makeatother
% Scale images if necessary, so that they will not overflow the page
% margins by default, and it is still possible to overwrite the defaults
% using explicit options in \includegraphics[width, height, ...]{}
\setkeys{Gin}{width=\maxwidth,height=\maxheight,keepaspectratio}
% Set default figure placement to htbp
\makeatletter
\def\fps@figure{htbp}
\makeatother
\setlength{\emergencystretch}{3em} % prevent overfull lines
\providecommand{\tightlist}{%
  \setlength{\itemsep}{0pt}\setlength{\parskip}{0pt}}
\setcounter{secnumdepth}{5}
\usepackage{booktabs}
\usepackage[]{natbib}
\bibliographystyle{apalike}

\title{2021 Third Year Phage Hunters}
\author{Ben Temperton}
\date{2021-05-10}

\begin{document}
\maketitle

{
\setcounter{tocdepth}{1}
\tableofcontents
}
\newpage

\hypertarget{welcome-phage-hunters}{%
\chapter*{Welcome Phage Hunters!}\label{welcome-phage-hunters}}
\addcontentsline{toc}{chapter}{Welcome Phage Hunters!}

\hypertarget{introduction}{%
\chapter*{Introduction}\label{introduction}}
\addcontentsline{toc}{chapter}{Introduction}

\hypertarget{lab-times}{%
\section*{Lab times}\label{lab-times}}
\addcontentsline{toc}{section}{Lab times}

Lab sessions will run from 0930-1230 each day in GP108. The kick-off meeting (Monday 17th May) where students will be handed a sampling pack will take place in a lecture theatre to be determined.

\hypertarget{phage-enrichment}{%
\chapter{Phage Enrichment}\label{phage-enrichment}}

In total, we will have 15 students isolating phages for the CPL. Each student will be assigned a pathogen and will use that to isolate phages from samples crowdsourced by their peers. The first week is spent enriching for phages from their samples. In total there will be 47 samples (plus one negative ctrl).

\hypertarget{monday-17th-may}{%
\section{Monday 17th May}\label{monday-17th-may}}

\begin{itemize}
\tightlist
\item
  Kick off meeting + students given sample packs, with samples to be taken next morning.
\end{itemize}

BT will give a presentation on the work and the steps involved in phage isolation.

\hypertarget{materials-required}{%
\subsection{Materials required}\label{materials-required}}

\begin{itemize}
\tightlist
\item
  36 sample jars
\item
  12 lab pens
\end{itemize}

\hypertarget{tuesday-18th-may}{%
\section{Tuesday 18th May}\label{tuesday-18th-may}}

\begin{itemize}
\tightlist
\item
  Students return to lab with samples
\item
  Samples are transferred to 50 mL falcon tube + centrifuged for 30 mins @ 10000 x g
\item
  Samples are filtered through 0.2 µm syringe filters into fresh 50 mL Falcon tube, then aliquoted into 15 1.5 mL lo-bind microcentrifuge tubes
\item
  Samples are recorded on class shared spreadsheet
\item
  Samples stored O/N at 4 °C.
\end{itemize}

\hypertarget{materials-required-1}{%
\subsection{Materials required}\label{materials-required-1}}

\begin{itemize}
\tightlist
\item
  6 x 50 mL Falcon tube x 15 students (90 total)
\item
  3 x 25 mL syringe with luer lock (45 total)
\item
  3 x 0.2 µm syrige filter (45 total)
\item
  3 x 15 1.5 mL microcentrifuge tubes (675 total)
\item
  microcentrifuge rack
\item
  Centrifuge capable of 10,000 x g for 50 mL Falcon tubes
\end{itemize}

\hypertarget{wednedsay-19th-may}{%
\section{Wednedsay 19th May}\label{wednedsay-19th-may}}

\begin{itemize}
\tightlist
\item
  Students assigned a pathogen
\item
  Students provided a set of samples
\item
  Students aliquot 900 µL of each sample into each well of a 96-well deep well plate
\item
  Students aliquot 500 µL of 3x LB + 30 mM \(MgCl_{2}\) + 30 mM \(CaCl_{2}\) into each well
\item
  Students aliquot 100 µL of O/N pathogen culture into each well.
\item
  Cover and incubate overnight at 30 °C on an orbital shaker.
\item
  Students set up fresh O/N culture of pathogen
\end{itemize}

\hypertarget{materials-required-2}{%
\subsection{Materials required}\label{materials-required-2}}

\begin{itemize}
\tightlist
\item
  10 mL of O/N pathogen culture
\item
  50 mL of 3x LB + 30 mM \(MgCl_{2}\) + 30 mM \(CaCl_{2}\) (1L total)
\item
  Deep well plate (15 total)
\item
  PCR film (15 total)
\item
  2 sterilins with 10 mL LB + 10 mM \(MgCl_{2}\) + 10 mM \(CaCl_{2}\) (30 total)
\end{itemize}

\hypertarget{thursday-20th-may}{%
\section{Thursday 20th May}\label{thursday-20th-may}}

\begin{itemize}
\tightlist
\item
  Students transfer 200 µL from each well into a 0.2 µm filter plate atop a regular, sterile 96 well plate.
\item
  Plate is spun at 900 x g for 4 mins to transfer filtrate to bottom plate
\item
  Into a fresh 200 µL 96-well plate, students add 190 µL of LB + 10 mM \(MgCl_{2}\) + 10 mM \(CaCl_{2}\) per well
\item
  5 µL of phage lysate from bottom plate is added to each well
\item
  10 µL of O/N host culture added to each well and plate sealed
\item
  Placed in an incubator O/N at 37 °C.
\item
  Students set up fresh O/N culture of pathogen
\end{itemize}

\hypertarget{materials-required-3}{%
\subsection{Materials required}\label{materials-required-3}}

\begin{itemize}
\tightlist
\item
  0.45 µm filter plate (Merck MSHAS4510) (15 total)
\item
  2 x sterile 200 µL 96-well plate (30 total)
\item
  PCR film (15 total)
\item
  2 sterilins with 10 mL LB + 10 mM \(MgCl_{2}\) + 10 mM \(CaCl_{2}\) (30 total)
\end{itemize}

\hypertarget{friday-21st-may}{%
\section{Friday 21st May}\label{friday-21st-may}}

\begin{itemize}
\tightlist
\item
  Students transfer 200 µL from each well into a 0.2 µm filter plate atop a regular, sterile 96 well plate.
\item
  Plate is spun at 900 x g for 4 mins to transfer filtrate to bottom plate
\item
  Bottom plate is covered with PCR film and stored until Monday at 4 °C.
\item
  Students streak pathogen onto agar plate for growth over weekend.
\end{itemize}

\hypertarget{materials-required-4}{%
\subsection{Materials required}\label{materials-required-4}}

\begin{itemize}
\tightlist
\item
  0.45 µm filter plate (15 total)
\item
  1 x sterile 200 µL 96-well plate (15 total)
\item
  PCR film (15 total)
\item
  2 x LB agar plates for bacterial growth
\item
  Inoculating loops
\end{itemize}

\hypertarget{spot-assays-and-purification}{%
\chapter{Spot Assays and Purification}\label{spot-assays-and-purification}}

In the second week, the students will be performing spot assays and phage purification.

\hypertarget{monday-24th-may}{%
\section{Monday 24th May}\label{monday-24th-may}}

Students will set up four top agar spot plates, with 12 samples per plate (w. neg. ctrl).

\begin{itemize}
\tightlist
\item
  Students will mark plates into quadrants
\item
  Students will add 1 mL of host culture at \$OD\_\{600\} of 0.6 to 3 mL of molten top agar (0.6\% agar in LB + 10 mM \(MgCl_{2}\) + 10 mM \(CaCl_{2}\)), vortex briefly and then pour onto bottom plate
\item
  Once the plates are set, students will spot 5 µL of phage lysate from each sample at 3 per quadrant
\item
  Plates will be incubated O/N at 37 °C for plaque formation
\end{itemize}

\hypertarget{materials-required-5}{%
\subsection{Materials required}\label{materials-required-5}}

\begin{itemize}
\tightlist
\item
  10 mL pathogen culture
\item
  5 glass test tubes with lids containing 3 mL molten agar (75 total)
\item
  5 petri dishes with bottom agar (75 total)
\item
  water bath (or heat blocks if the tubes will fit in).
\end{itemize}

\hypertarget{tuesday-25th-may}{%
\section{Tuesday 25th May}\label{tuesday-25th-may}}

Students will pick plaques and perform first round of O/N purification steps. We are going to assume a maximum of 3 phages per student.

\begin{itemize}
\tightlist
\item
  Students will take a photo of their plates to record plaque morphology
\item
  For each plaque, students will core the plaque using a pipette tip and place it in 200 µL of SM buffer in a microcentrifuge tube, before vortexing briefly.
\item
  Students will then set up a dilution series from \(10^{0}\) to \(10^{-11}\) in SM Buffer in a 96 well plate (one row per phage)
\item
  Using the same method as the spot assay, students will spot the 12 dilutions onto a single plate (so one plate per phage).
\item
  Plates will be left O/N for plaques to develop
\end{itemize}

\hypertarget{materials-required-6}{%
\subsection{Materials required}\label{materials-required-6}}

\begin{itemize}
\tightlist
\item
  10 mL pathogen culture
\item
  3 glass test tubes with lids containing 3 mL molten agar (45 total)
\item
  3 petri dishes with bottom agar (45 total)
\item
  water bath (or heat blocks if the tubes will fit in).
\item
  1 sterile 96 well plate (15 total)
\item
  3 lo-bind microcentrifuge tubes (45 total)
\end{itemize}

\hypertarget{wednesday-26th-may}{%
\section{Wednesday 26th May}\label{wednesday-26th-may}}

From each plate, students will pick plaque from the biggest dilution (fewest phages) and perform second round of O/N purification. We are going to assume a maximum of 3 phages per student.

\begin{itemize}
\tightlist
\item
  Students will take a photo of their plates to record plaque morphology
\item
  For each plaque, students will core the plaque using a pipette tip and place it in 200 µL of SM buffer in a microcentrifuge tube, before vortexing briefly.
\item
  Students will then set up a dilution series from \(10^{0}\) to \(10^{-11}\) in SM Buffer in a 96 well plate (one row per phage)
\item
  Using the same method as the spot assay, students will spot the 12 dilutions onto a single plate (so one plate per phage).
\item
  Plates will be left O/N for plaques to develop
\end{itemize}

\hypertarget{materials-required-7}{%
\subsection{Materials required}\label{materials-required-7}}

\begin{itemize}
\tightlist
\item
  10 mL pathogen culture
\item
  3 glass test tubes with lids containing 3 mL molten agar (45 total)
\item
  3 petri dishes with bottom agar (45 total)
\item
  water bath (or heat blocks if the tubes will fit in).
\item
  1 sterile 96 well plate (15 total)
\item
  3 lo-bind microcentrifuge tubes (45 total)
\end{itemize}

\hypertarget{thursday-27th-may}{%
\section{Thursday 27th May}\label{thursday-27th-may}}

From each plate, students will pick plaque from the biggest dilution (fewest phages) and perform third round of O/N purification. We are going to assume a maximum of 3 phages per student.

\begin{itemize}
\tightlist
\item
  Students will take a photo of their plates to record plaque morphology
\item
  For each plaque, students will core the plaque using a pipette tip and place it in 200 µL of SM buffer in a microcentrifuge tube, before vortexing briefly.
\item
  Students will then set up a dilution series from \(10^{0}\) to \(10^{-11}\) in SM Buffer in a 96 well plate (one row per phage)
\item
  Using the same method as the spot assay, students will spot the 12 dilutions onto a single plate (so one plate per phage).
\item
  Plates will be left O/N for plaques to develop
\item
  Students will set up O/N cultures of host
\end{itemize}

\hypertarget{materials-required-8}{%
\subsection{Materials required}\label{materials-required-8}}

\begin{itemize}
\tightlist
\item
  10 mL pathogen culture
\item
  3 glass test tubes with lids containing 3 mL molten agar (45 total)
\item
  3 petri dishes with bottom agar (45 total)
\item
  water bath (or heat blocks if the tubes will fit in).
\item
  1 sterile 96 well plate (15 total)
\item
  3 lo-bind microcentrifuge tubes (45 total)
\item
  2 sterilins containing 10 mL LB + 10 mM \(MgCl_{2}\) + 10 mM \(CaCl_{2}\) (30 total)
\end{itemize}

\hypertarget{friday-28th-may}{%
\section{Friday 28th May}\label{friday-28th-may}}

From each plate, students will pick plaque from the biggest dilution (fewest phages) and bulk it up on hosts overnight.

\begin{itemize}
\tightlist
\item
  Students will take a photo of their plates to record plaque morphology. These will also be used to determine approximate PFU counts.
\item
  Students will pick plaque from the biggest dilution (fewest phages) and add it to 50 mL of LB + 10 mM \(MgCl_{2}\) + 10 mM \(CaCl_{2}\), amended with 1 mL of O/N host culture.
\item
  Cultures will be grown O/N at 30 °C
\end{itemize}

The next day (Saturday), BT and team will go in and centrifuge the samples down and prepare for each phage 1 x 50 mL tubes of lysate, filtered through a 0.2 µm syringe filter. We will prepare 3 x 2 mL phage stocks in acid washed, autoclaved amber glass vials.

\textbf{200 µL of each phage filtrate will be provided to the imaging centre.}

\hypertarget{materials-required-9}{%
\subsection{Materials required}\label{materials-required-9}}

\begin{itemize}
\tightlist
\item
  3 x 50 mL LB + 10 mM \(MgCl_{2}\) + 10 mM \(CaCl_{2}\) (2L total)
\item
  3 x 50 mL falcon tubes (45 total)
\item
  3 x 0.2 µm syringe filter (45 total)
\item
  3 x 25 mL syringe with luer lock (45 total).
\item
  3 x 2 mL amber glass vials
\end{itemize}

\hypertarget{phage-purification-and-dna-extraction}{%
\chapter{Phage Purification and DNA extraction}\label{phage-purification-and-dna-extraction}}

In the third week, the students will perform DNA extractions on phages

\hypertarget{tuesday-1st-june}{%
\section{Tuesday 1st June}\label{tuesday-1st-june}}

Students will test their lysate for estimating PFU and begin phage DNA precipitation

\begin{itemize}
\tightlist
\item
  Students will prepare a dilution series of their phages as before and spot them in triplicate across 3 plates
\item
  Students will transfer 30 mL of lysate to a fresh Falcon tube
\item
  Students will add 15 µL of nuclease solution and incubate at 37 °C for 30 mins
\item
  Students will then add 15 mL of precipitant solution to each tube and mix gently by inversion
\item
  Samples will be incubated at 4 °C O/N
\end{itemize}

\hypertarget{materials-required-10}{%
\subsection{Materials required}\label{materials-required-10}}

\begin{itemize}
\tightlist
\item
  10 mL pathogen culture
\item
  3 glass test tubes with lids containing 3 mL molten agar (45 total)
\item
  3 petri dishes with bottom agar (45 total)
\item
  water bath (or heat blocks if the tubes will fit in).
\item
  1 sterile 96 well plate (15 total)
\item
  3 x Falcon tube (45 total)
\item
  50 µL of nuclease solution (750 µL total)
\item
  50 mL of precipitant solution (750 mL total)
\end{itemize}

\hypertarget{wednesday-2nd-june}{%
\section{Wednesday 2nd June}\label{wednesday-2nd-june}}

Students will extract DNA using the Promega Wizard kit using \href{https://cpt.tamu.edu/wordpress/wp-content/uploads/2011/12/Phage-DNA-extraction-modified-Wizard-method-07-12-2011.pdf}{this protocol}

Depending on how many phages we isolate, we are anticipating sending two per student for sequencing, with the remaining extractions kept back for future analyses.
\#\#\# Materials required
* 3 x 500 µL resuspension buffer (5 mM \(MgSO_{4}\)) (25 mL total)
* 2 x Promega Wizard Kits.

\hypertarget{thursday-3rd-june}{%
\section{Thursday 3rd June}\label{thursday-3rd-june}}

Students will perform killing efficiency assays of their phages.

\textbf{Group photo!}

\hypertarget{friday-4th-june}{%
\section{Friday 4th June}\label{friday-4th-june}}

Christian and Lauren will provide an overview on using the tools via Zoom.

Two videos from the CPT also exist for this purpose:

\begin{itemize}
\tightlist
\item
  \href{https://t.co/F9Gv0rCHhS}{Structural Pipeline Video}
\item
  \href{https://t.co/5veyKIswWv}{Functional Pipeline Video}
\end{itemize}

\hypertarget{genome-annotation}{%
\chapter{Genome Annotation}\label{genome-annotation}}

The students will use the Centre for Phage Technology Galaxy Phage tool to annotate their genomes

\hypertarget{monday-7th-june}{%
\section{Monday 7th June}\label{monday-7th-june}}

Sequencing data comes back from the sequencing centre. BT performs assembly and read mapping and gives data to the students

\hypertarget{tuesdsay-8th---friday-11th-june}{%
\section{Tuesdsay 8th - Friday 11th June}\label{tuesdsay-8th---friday-11th-june}}

Students will embark on analysing their genomes. This will include:

\begin{enumerate}
\def\labelenumi{\arabic{enumi}.}
\tightlist
\item
  Classification using \href{https://www.genome.jp/viptree/}{VIPTree}
\item
  Identification of novel species and genera through \href{http://rhea.icbm.uni-oldenburg.de/VIRIDIC/}{VIRIDIC}
\end{enumerate}

\hypertarget{recipes-for-reagents}{%
\chapter*{Recipes for reagents}\label{recipes-for-reagents}}
\addcontentsline{toc}{chapter}{Recipes for reagents}

\hypertarget{precipitant-solution}{%
\section{Precipitant solution}\label{precipitant-solution}}

This is a ready-mixed 30\% w/v PEG-8000, 3 M NaCl solution for adding to phage lysate in a 1:2 ratio precipitant:lysate (10\%PEG-8000, 1 M NaCl final conc.)

\begin{enumerate}
\def\labelenumi{\arabic{enumi}.}
\tightlist
\item
  In a sterilized 500 mL bottle add 330 mL of autoclaved MilliQ and 105 g of NaCl and dissolve.
\item
  Add 180g of PEG8000, cap bottle and shake.
\item
  Incubate bottle in a 60 °C waterbath for 3 hours, shaking occasionally
\item
  Remove and let cool to RT, shaking occasionally
\item
  Add autoclaved MilliQ to 600 mL and store at RT.
\end{enumerate}

\hypertarget{beef-resuspension-solution}{%
\section{Beef resuspension solution}\label{beef-resuspension-solution}}

This is based on the recipe from \citep{Williamson2003-fu}. For each sample of 25g:

\begin{enumerate}
\def\labelenumi{\arabic{enumi}.}
\tightlist
\item
  To a 500 mL Duran add the following:
\end{enumerate}

\begin{itemize}
\tightlist
\item
  25g of beef extract
\item
  3.35g of \(Na_{2}HPO_{4}\cdot7H_{2}O\)
\item
  0.3g of citric acid
\end{itemize}

\begin{enumerate}
\def\labelenumi{\arabic{enumi}.}
\setcounter{enumi}{1}
\tightlist
\item
  Add 250 mL of autoclaved MQ
\item
  Add a stirrer bar
\item
  Mix until completely dissolved and adjust pH to 7.2
\end{enumerate}

\hypertarget{protocols}{%
\chapter*{Protocols}\label{protocols}}
\addcontentsline{toc}{chapter}{Protocols}

\hypertarget{preparing-phages-from-environmental-samples}{%
\section{Preparing phages from environmental samples}\label{preparing-phages-from-environmental-samples}}

\textbf{Note: Due to the need to centrifuge samples in a fixed rotor, you can only process \emph{eight} samples at any one time}

Please read all associated Risk Assessments related to the pathogens of interest before proceeding.

\hypertarget{solid-samples}{%
\subsection{Solid Samples}\label{solid-samples}}

This method is adapted from \citep{Guzman2007-na}. The proteins in the beef extract help phages desorb from solid material. Proteinaceous substances within the beef extract and those that desorb with the phages can interfere with downstream molecular analyses on large samples (e.g.~PCR) \citep{Goller2020-wx}, but for phage isolation, this isn't a problem (and beef extract is more cost effective than using PBS + BSA).

\begin{enumerate}
\def\labelenumi{\arabic{enumi}.}
\tightlist
\item
  Aliquot 10-25g of sample into a 250 mL Duran, with a magnetic stirrer
\item
  Add 1:10 w/v \protect\hyperlink{beef-resuspension-solution}{10\% beef resuspension solution (pH 7.2)}
\item
  Homogenise the sample by magnetic stirring for 20 mins at RT with sufficient speed to develop a vortex (500-900 rpm)
\item
  Transfer to 50 mL falcon tubes or larger vessels and centrifuge at 4080 xg for 10 mins at 4 °C. This stage is just to get rid of the bulk material, so can be done in the benchtop centrifuge in GP211.
\item
  Pass the supernatant through a Pluriselect 40 µm mesh filter into a fresh 50 mL Falcon tube to the 40 mL mark. Draw an arrow on the lid and mark the side where the pellet is going to be.
\item
  Centrifuge at 10,000 xg for 30 mins at 4 °C. This requires the ThermoScientific Multifuge X3R on the 4th floor.
\item
  In a flow hood, filter through a 0.2 µm PES filter into an acid-washed and autoclaved 20 mL amber glass vial (Fisher 11503552). You may have to re-centrifuge if it's really clogging the filter. Use a luer-lock syringe and filter for this to avoid it popping off and spraying everywhere! You may need to use a second filter if it starts to clog.
\item
  Label the glass vial with the \href{https://what3words.com/}{What3Words} label. Also on the side, write the date it was processed and your name.
\item
  Store at 4 °C in the dark in the CPL sample box
  \newpage
\end{enumerate}

\hypertarget{liquid-samples}{%
\subsection{Liquid Samples}\label{liquid-samples}}

\begin{enumerate}
\def\labelenumi{\arabic{enumi}.}
\tightlist
\item
  Aliquot sample into a 50 mL Falcon tube. Draw an arrow on the lid and mark the side where the pellet is going to be.
\item
  Top up to 45 mL with pH 7.5 buffer if necessary
\item
  Centrifuge at 10,000 xg for 30 mins at 4 °C. This requires the ThermoScientific Multifuge X3R on the 4th floor.
\item
  Pour supernatant into 20 mL syringe fitted with 0.2 µm PES filter
\item
  In a flow hood, filter through a 0.2 µm PES filter into an acid-washed and autoclaved 20 mL amber glass vial (Fisher 11503552). You may have to re-centrifuge if it's really clogging the filter. Use a luer-lock syringe and filter for this to avoid it popping off and spraying everywhere!
\item
  Label the glass vial with the \href{https://what3words.com/}{What3Words} label. Also on the side, write the date it was processed and your name.
\item
  Store at 4 °C in the dark in the CPL sample box
  \newpage
\end{enumerate}

\hypertarget{initial-phage-enrichment-using-2.2-ml-deep-well-plates-or-7-ml-sterilins}{%
\section{1. Initial phage enrichment (using 2.2 mL deep well plates or 7 mL sterilins)}\label{initial-phage-enrichment-using-2.2-ml-deep-well-plates-or-7-ml-sterilins}}

This method is based off this work \citep{Olsen2020-dh} and allows for the screening of one or more host strains against up to 94 samples (plus a blank and a positive control). Briefly, into each well we are adding LB medium into filtered sample that contains phages, then adding in an overnight culture of the pathogen of interest. If there are phages in the sample that infect the host, they will replicate and thus increase in abundance within the well. Subsequent steps then purify these enriched phages.

For a negative control, you want to use a sample that has no phages in it. DI water is recommended for this. You also want to run a positive control (to show that phages can be amplified and plated). For this, you can use a phage lysate that has been previously isolated on the host of interest. Spike 10 µL of previous phage lysate into 1340 µL of DI water as your positive control.

If you do not have a phage isolated on the pathogen of interest, it is recommended that you use a positive control consisting of 10 µL of T4 lysate into 1340 µL of DI. You will then need a well on your plate that contains an enrichment of \emph{E. coli} BW25113. You'll also need to include this in subsequent plaque assay steps.

\begin{enumerate}
\def\labelenumi{\arabic{enumi}.}
\tightlist
\item
  Make an O/N culture of your host of interest in LB medium at 37 °C on an orbital shaker. The required volume per host is 110 µL \(\times\) the number of samples being tested, plus one negative control and one positive. So for 10 samples plus controls, you will need \textasciitilde2 mL of host culture.
\item
  Make a 40 mL solution of 0.25 M \(CaCl_2\) and 0.25 M \(MgCl_2\) as follows:
\end{enumerate}

\begin{itemize}
\tightlist
\item
  10 mL 1M \(CaCl_2\)
\item
  10 mL 1M \(MgCl_2\)
\item
  20 mL DI water
\item
  Vortex and filter sterilize through a 0.22 µm syringe filter.
\end{itemize}

\begin{enumerate}
\def\labelenumi{\arabic{enumi}.}
\setcounter{enumi}{2}
\tightlist
\item
  The next day, unwrap an autoclaved 2.2 ml deep well 96-well plate (make sure you have the right ones! The wells are square. The ones with round wells don't hold enough liquid) in a flow hood.
\item
  Place 1350 µL of a 0.45 µm filtered sample into each well. For every strain of host, you need a negative control well (1350 µL of DI water). It's worth thinking about the final spot plaque assays when laying out the plate. It's easier to have a whole row or a whole column with a single host strain, as that way, you can spot an entire row/column onto the same plate, rather than having to cherry-pick samples.
\item
  To each well, add 80 µL of a mixture of 0.25 M \(CaCl_2\) and 0.25 M \(MgCl_2\)
\item
  To each well, add in 100 µL of an O/N culture of host
\item
  To each well, add in 450 µL of 4.4x concentration LB (44g of LB in 400 mL)
\item
  Pipette up and down to mix
\item
  Cover with a plate film.
\item
  Incubate overnight at 30 °C on an orbital shaker at 200 rpm.
\item
  Set up another O/N culture of the hosts for the next day.
\end{enumerate}

  \bibliography{book.bib,packages.bib}

\end{document}
